% !TeX root = ../report.tex
% !TeX spellcheck = en-US
% !TeX encoding = UTF-8

\chapter{CONCLUSION}\label{chap:conclusion}
\glsresetall

To control something you have to know it. A controller for an autonomous operating crawler, encompasses more than
implementing a simple state-machine, which tells it to go left or right. It starts with knowing how physics of its 
environment govern the state change of such a crawler. Only from this point can you go forth and actually determine how 
such as controller should behave.

\medskip
\noindent \textit{This thesis answers how such a controller might behave for an Archimedes driven dredge crawler.}
\medskip\newline

The crawler will operate in a challenging environment, namely underwater. The dampening effects of water are 
notorious. This is limiting factor in the choice for communication methods. Leaving only an umbilical as a viable 
choice. The greater challenge of this infamous dampening effect, is that is creates a \gls{acr-GPS} deprived 
environment. Where land based vehicles have access to a mature, accurate and fast \gls{acr-GPS} network, underwater 
vessel need to resort to dead-reckoning techniques. Which are well-known for their growing error.

Sensor fusion can be a great tool to battle this effect. A distinction is made for sensors that aid in describing the 
state of the vehicle and those that describe the environment. Were detected landmarks can act as ground truths, 
anchoring the otherwise unbound error. State sensing is usually done with \glspl{acr-IMU}, consisting of an 
accelerometer, gyroscope and magnetometer. These are also of use in the case of a crawler. Coupled with a pressure 
sensor, which measures the water column above the crawler, helps in an accurate estimation of the \gls(sym-z) 
position. This accuracy of the estimate will however degrade when dredging operations disturb the soil, which mixes 
with the water and alter the density.

Localization under uncertainty can be refined with the use of a Kalman filter, which in the case of the crawler is an
\gls{acr-UKF}. Due to the inherent non-linear behavior of the kinematic model. A Kalman filter estimate where the
crawler should be according to a state transition matrix \gls{sym-F} and a input vector \gls{sym-u_k}. This estimate 
is updated with the measured values and a calculated error covariance matrix \gls{sym-P_k}. Which will become more 
accurate over time. Filtering out noise and inaccuracies introduced due to sensor characteristics. If the state 
transistion model is accurate, the Kalman filter will have a great impact on the quality of the overal performance of
the controller.

A crawler designed for dredging operations, has to cover an certain area, which might be unknown and can be changeing
in time. Obstacles --- such as ships --- might move, and the crawler has to take that into account. These challenges 
are known as \gls{gls-on-line} \gls{acr-CPP} problems. Three categories are explored in this thesis. Morse-based 
cellular decompositions, landmark-based topological coverage and grid-based methods. These coverage algorithms are in
effect ``divide and conquer'' strategies, and decompose a complex space in sub-regions. Were each sub-regions is 
covered before moving to the next sub-region. After careful consideration a choice is made for the implementation of
a \gls{acr-topBCD}. This is a landmark-based topological type of strategy which borrows the \gls{acr-BCD} approach from 
the Morse-based cellular decomposition strategy. \gls{acr-BCD} literally means ``way-of-the-ox'', meaning that the 
crawler will go back and forth in the sub-regions, as an ox might plow a field.

Implementation of the \gls{acr-UKF} stands or fall with an accurate kinematic model. Estimating where the crawler 
will be requires a complete model of the drive train. Which is model on first-principle, where the output of each 
component represent an energy transfer to the next. This drive-train actuates two rotating Archimedes screws, the 
terramechanic model describes how the rotation is actually transformed in a translation of the crawler on the soil 
bed. It proposes a novel method of estimating slippage by measuring an change in generated torque. Explaining that 
the blade of the Archimedes screw act as shovel when slippage occurs, resulting in an increase of torque.

A controller framework ``ohCaptain'' is written in C++, and released under the LGPL v3.0 license. It can run on 
\gls{acr-SBC}, ranging from expensive dedicated industrial devices, to cheap \glspl{acr-Rpi} or \glspl{acr-BBB}. 
Allowing future prototype to benefit from this projecty. Resulting in lower costs and shorter lead times. The 
framework is specifically developed for maritime operations and is set-up in a modulair way. Allow for future 
expansion and specialized algortihms.

The research and design choices have been validated with the help of Project Chrono, a multi body physics simulation 
engine. This engine has dedicated modules for vehicles and terramechanics; It was however lacking a sensor simulation
module. Which was needed to benchmark the performance of the \gls{acr-UKF}. An extension module was created and 
released under the MIT license, allowing for the simulation of realistic sensor measurements. Taking into account 
noise, discretization due to digitization, and transformation of the signal, simulation drift and hard-/soft-iron 
effects.

Simulating use-case 2, which is a coverage maintenance task of an existing harbor in Bruinisse the Netherlands, show 
that the chosen \gls{acr-topBCD} strategy is executed in a viable way. Dividing the harbor into three sub-regions, each
completely covered with the path planner. The simulation gives also an insight into the performance of the propossed 
\gls{acr-UKF}. Comparing its performance to with a simple \gls{acr-PID} controller. It shows a huge difference, were 
the \gls{acr-PID} conroller will derail quite early in its travels. Due to an inaccurate position estimation. The 
\gls{acr-UKF} controller is able to fully cover the harbor. 

\medskip\newline
\noindent The controller shows great potential, but as it is with all research and development papers: ``Further 
research is needed.''

