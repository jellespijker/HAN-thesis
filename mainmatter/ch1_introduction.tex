% !TeX root = ../report.tex
% !TeX spellcheck = en-US
% !TeX encoding = UTF-8


\chapter{INTRODUCTION}\label{chap:introduction}

The last decenium has shown a spurt in the acceptance and adaptation of machine learning. Smart controllers
capable of performing complex tasks without interference of human operator. These range from ``smart'' cars to home
owners whom integrate cheap~\glspl{acr-SBC}, such as a \gls{acr-Rpi}, into their home. This general acceptance and an
abundance of cheap available devices, has given a real boost to the development of machine learning and autonomous 
operating machines.

\gls{acr-IHC} is investigating on a strategic level how to best place this technology in their product portfolio. One
of the key parameters is knowing the~\gls{acr-TRL}. Multiple business units within~\gls{acr-IHC} are developing
prototypes of autonomous operating machines. Ranging from catamarans to deep sea mining crawlers. A lot of these 
explorations are usually small side projects. With short leads and limited budgets. All give valuable insights 
and are of import in developing a familiarization with this technology.

\gls{acr-MTI} is one of these business units exploring the technological challenges that of various autonomous devices.
They're the reseach and development department within~\gls{acr-IHC}. They adopted an Archimedes driven crawler from 
one of their sibling business units in 2016. This crawler was unfinished at the time.

This chapter will first specify three use-cases, specified in the project assignment, in which an crawler must 
operate. It then describes basic principles, applications and tools relevant for these use cases.


\section{USE CASES}\label{sec:usecases}
The use case below are determined by ir. F. Hofstra, these cases are expected to be valid and realistic. Keeping in 
mind their marketability. These cases will determine the needed functionality for an crawler and stand at the basis 
for the controller design.

\subsection{ARBITRARY SHAPED SPACE}\label{sec:usecase1}
An crawler is placed in a predefined arbitrary shaped space, not too complex, with an area of \( 3500 
\si{\square\metre} \). The shape of this space is set, but the movement pattern is unrestricted. The crawler has to 
remove a layer with a depth of \( 5 \si{\cm} \). The controller has to determine an optimal path with the least 
amount of time or the shortest path. This can be coupled with learning capabilities and an analyze capacity. At a 
later time additional constrains can be added which keep in mind the deployment location of a flexible 
\gls{gls-dredgeline} and an \gls{gls-umbilical}.

\subsection{MARINA AQUA DELTA}\label{sec:usecase2}
The crawler operates in a predefined space with obstacles, not every obstacles is known. The actual location is 
\href{https://www.google.nl/maps/place/Jachthaven+Bruinisse/@51.6712838,4.0824101,
15z/data=!4m2!3m1!1s0x0:0x9c840ab80bde39c8}{marina Aqua Delta} located in Bruinisse, the Netherlands. The shape of 
this location is set but the movement pattern is unrestricted. An crawler has to remove a layer with a depth of \( 5 
\si{\cm} \). The controller has to determine an optimal path with the least amount of time or the shortest path. This
can be coupled with learning capabilities and an analyze capacity. The marina has enough depth for the crawler to 
move underneath the scaffolding.
No consideration has to be made for a flexible \gls{gls-dredgeline} and a \gls{gls-umbilical}. These conditions are 
introduced at a later stage.

\subsection{THREE GORGES DAM}\label{sec:usecase3}
An crawler operates in a predefined space with obstacles, not every location of those obstacles is known. The 
predefined space is located at the foot of \href{https://www.google.nl/maps/place/Three+Gorges+Dam/@30.8263416,111
.0118356,16z/data=!4m2!3m1!1s0x368476d5e9a340d9:0xa017b4d524bd9d6c}{three Gorges dam}. Silt is deposited at the foot 
of this dam, due to natural occurring \gls{gls-erosion} and \gls{gls-sedimentation}. The accumulation of 
\gls{gls-silt} can be controlled by dredging localized pits. Which in turn create locations with a lower density. 
This induces a gravity driven density current towards those locations. The crawler has to maintain an average nominal
depth with a certain silt deposit rate.


\section{ARCHIMEDES DRIVEN CRAWLER}
