% !TeX root = ../report.tex
% !TeX spellcheck = en-US
% !TeX encoding = UTF-8


\chapter{INTRODUCTION}\label{chap:introduction}

The last decenia has shown a spurt in the acceptance and adaptation of machine learning. Smart controllers
capable of performing complex tasks without interference of human operator. These range from ``smart'' cars to 
home-owners whom integrate cheap~\glspl{acr-SBC}, such as a \gls{acr-Rpi}, into their home. This general acceptance 
and an abundance of cheap available devices, has given a real boost to the development of machine learning and 
autonomous operating machines.


\section{BACKGROUND}\label{sec:background}

\gls{acr-IHC} is investigating on a strategic level how to best place this technology in their product portfolio. One
of the key parameters is knowing the~\gls{acr-TRL}. Multiple business units within~\gls{acr-IHC} are developing
prototypes of autonomous operating machines. Ranging from catamarans to deep sea mining crawlers. A lot of these
explorations are usually small side projects. With short leads and limited budgets. All give valuable insights
and are of import in developing a familiarization with this technology.

\gls{acr-MTI} is one of these business units exploring the technological challenges that of various autonomous devices.
They're the research and development department within~\gls{acr-IHC}. They adopted an Archimedes driven crawler from
one of their sibling business units. This crawler was unfinished at the time and ~\gls{acr-MTI} saw it as a great
opportunity to learn.

This thesis discusses the design of a controller, for autonomous maritime operations; First and foremost for the
operations of an Archimedes driven dredging crawler; In addition, it will take into account low budget considerations
and rapid prototype requirements. Such that it can serve as a stepping stone for future autonomous operating maritime
vessels. The crawler in this document is a relative small dredging crawler, designed for maintenance tasks in
basins, harbors and around dams.


\section{USE CASES}\label{sec:usecases}
Three use cases have been specified by ir. F. Hofstra, these cases are valid and realistic scenarios. Keeping in
mind their marketability. These cases will determine the needed functionality for a crawler and stand at the basis
for the controller design.

\subsection{ARBITRARY SHAPED SPACE}\label{sec:usecase1}
An crawler is placed in a predefined arbitrary shaped space, not too complex, with an area of \( 3500
\si{\square\metre} \). The shape of this space is set, but the movement pattern is unrestricted. The crawler has to
remove a layer with a depth of \( 5 \si{\cm} \). The controller has to determine an optimal path with the least
amount of time or the shortest path. This can be coupled with learning capabilities and an analyze capacity. At a
later time additional constrains can be added which keep in mind the deployment location of a flexible
\gls{gls-dredgeline} and an \gls{gls-umbilical}.

\subsection{MARINA AQUA DELTA}\label{sec:usecase2}
The crawler operates in a predefined space with obstacles, not every obstacles is known. The actual location is
\href{https://www.google.nl/maps/place/Jachthaven+Bruinisse/@51.6712838,4.0824101,
15z/data=!4m2!3m1!1s0x0:0x9c840ab80bde39c8}{marina Aqua Delta} located in Bruinisse, the Netherlands. The shape of
this location is set but the movement pattern is unrestricted. An crawler has to remove a layer with a depth of \( 5
\si{\cm} \). The controller has to determine an optimal path with the least amount of time or the shortest path. This
can be coupled with learning capabilities and an analyze capacity. The marina has enough depth for the crawler to
move underneath the scaffolding. No consideration has to be made for a flexible \gls{gls-dredgeline} and a 
\gls{gls-umbilical}. These conditions can be introduced at a later stage.

\subsection{THREE GORGES DAM}\label{sec:usecase3}
An crawler operates in a predefined space with obstacles, not every location of those obstacles is known. The
predefined space is located at the foot of \href{https://www.google.nl/maps/place/Three+Gorges+Dam/@30.8263416,111
.0118356,16z/data=!4m2!3m1!1s0x368476d5e9a340d9:0xa017b4d524bd9d6c}{three Gorges dam}. Silt is deposited at the foot
of this dam, due to natural occurring \gls{gls-erosion} and \gls{gls-sedimentation}. The accumulation of
\gls{gls-silt} can be controlled by dredging localized pits. Which in turn create locations with a lower density.
This induces a gravity driven density current towards those locations. The crawler has to maintain an average nominal
depth with a certain silt deposit rate.

\section{STRUCTURE OF THIS DOCUMENT}\label{sec:structure of this document}

In Chapter~\ref{chap:dredgingprocess} a short introduction is made regarding dredging principles and their 
applications. Readers familiar with this subject can skip this this chapter. Chapter~\ref{chap:research} is a review 
of of the current state of technology with regards to autonomouse operations underwater. It investigates 
communication techniques, various sensing methods needed for localization, and describes how a position can be 
estimated in a \gls{acr-GPS} deprived environment. The final research section looks into various \gls{acr-CPP} 
algorithms, need for dredging operations.

With the gained knowledge from Chapter~\ref{chap:controller design} a design for a controller is proposed. A choice 
is made between different \gls{acr-CPP} strategies based on usefulness for the crawler and the needed peripherals 
laid out. It then describes the models for the state vector \gls{sym-x_k}, state transition matrix \gls{sym-F} and the
control vector \gls{sym-u_k}, needed for a \gls{acr-UKF}, which is used to determine the position under uncertainty. 
A framework ``ohCaptain'' is proposed, which is a C++ library, intended for controlling maritime autonomous vessel, 
with a focus on extensibility and rapid development.

The performance of the controller is validated using a multibody physics engine, called Project Chrono. The setup and
results are described in Chapter~\ref{chap:design validation}. An important part of measuring the performance of the 
controller, is grasping its ability to process noisy and transformed sensor signals. A feature, which was not present
in the physics engine's. An extension of Project Chrono, was written, which allows for realistic sensor simulations. 
The works are explained in this chapter as well. Finally the result are discussed.

\nodindent The final Chapter~\ref{chap:conclusion} looks back at the original goal and discusses how to proceed from 
here.