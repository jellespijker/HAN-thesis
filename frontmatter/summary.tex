% !TeX root = ../report.tex
% !TeX spellcheck = en-US
% !TeX encoding = UTF-8


\chapter{SUMMARY}\label{chap:sum}

This thesis centers around desiging, implementating, and validating a controller used in autonomous maritime 
operations. In particular operations of an Archimedes driven dredging crawler. It includes an extensive review of the
current state of technology with regards to underwater communications in Section~\ref{sec:underwater_communication}. 
Describing the dampening effects of water on the wireless communication, and explaining why \gls{acr-GPS} isn't 
suited for the work environment of a sub-merged crawler. Dead-reckoning is often used to overcome such limitations, 
but these bring their own challenges.

Dead-reckoning relies on state sensing sensors, which are described in Section~\ref{sec:sensors}. Commonly used 
\glspl{acr-IMU} are comprised of accelerometer, gyroscope, and magnetometer. It is not uncommon in underwater 
applications, to use a pressure sensor, to further limit the uncertainty of the depth estimate. 
Section~\ref{sec:pressure sensor} explains why this accuracy of might degrade during dredging operations, because 
these disturb the soil; creating a localized suspension with a different density, which results in an over estimation
of the crawlers depth.

It is imperative for underwater vessels to have a correct estimation of its position and pose. Since it is a 
\gls{acr-GPS} deprived environment a literature review has been performed how to minimize this uncertainty. A common 
method is the use of a \gls{acr-LQE} also known as a Kalman filter. A tried and practice method. It is used to 
control a real-world dynamic system and uses a fairly faithful replication of the true state of dynamics. It works by
estimating a state of the process, based on \textit{a priori} state \gls{sym-x_k}, which is transformed to \textit{a 
post priori} state \gls{sym-x_kt+1}. It is able to filter out white noise with a normal probability distribution. A 
simple linear example of a falling ball is described in Section~\ref{sec:basic Kalman filter}.

A literature review regarding different \gls{acr-CPP} strategies, is written in Section~\ref{sec:cpp}. It describes 
three
different categories: morse-based cellular decomposition, Landmark-based topological coverage
and grid-based methods. These methods all use a ``divide and conquer'' strategy. Dividing an unknown, dynamically
chancing environment in to sub-regions, based on certain distinct features. Section~\ref{sec:strategy decision}
compares the researched strategies and weighs them against each other. Settling on \gls{acr-topBCD}, a topological
coverage algorithm that borrows a \gls{acr-BCD} aspect from the morse-base strategies. It divides an area into
sub-regions based on topological landmark features, and it processes these sub-regions as an ox would plough the field.

Section~\ref{sec:peripherals} describes which pheripherals will be used, Section~\ref{sec:kalman design} explains the
design of the used Kalman filter, which is a \gls{acr-UKF}, due to the non-linear behaviour of the physics governing 
the kinematic and dynamic behaviour. A state representation for the crawler is described in Section~\ref{sec:state 
representation}. The state vector \gls{sym-x_k} describes position, velocity and pose and its derivative, in 
3-dimensions. Mathematical rotation are usually performed with a rotation matrix, this construct suffers from 
gimbal-lock and is computational ``heavy''. It is quite common to use quaternions for this kind of rotations. These 
are an extension to complex numbers. Instead of using one imaginary axis, they use three imaginary axes. The pose 
vector and its derivative are therefor represented as quaternions.

Section~\ref{sec:motion model} and Section~\ref{sec:soil dynamic model} describe how a control vector \gls{sym-u_k}
will translate to \textit{a post priori} state. It takes into account the characteristics of the drive-train, which is a
hydraulic system, actuated with an electrical motor. This drive-train rotates two mirrored Archimedes screw, and
transform this rotation in a trust forward. A novel proposal is made, estimating the slip between screw and soil in
the terramechanic model. The vanes of an Archimedes screw will act as a bulldozer, if the translation forward
is not proportional to the pitch of the vanes. A kinematic steering model of the crawler is based on a differential
drive, and is described in Section\ref{sec:steering model}.

Travel of the dredger has two States: normal travel and dredging. In normal travel the maximum velocity is limited by
the drive-train. During dredging the limiting factor is its ability of soil removal. Section~\ref{sec:dredge model}
determines the maximum allowable speed, governed by the flow of slurry in the dredging system. Resulting in maximum
speed of \SI{155}{\metre\per\hour} at an optimal production rate of \SI{140}{\cubic\metre\per\hour}.

A C++ controller framework named ``ohCaptain'' is specifically written for this project. The description of this
framework is provide in Section~\ref{sec:controller}. It's primarily intended for the control of autonomous maritime 
vessels. It is based on an actor-model design pattern, where each actor is responsible for its own
tasks, which is executed asynchronous; either on the same controller or on a cluster of controllers. The actor
responsible for the big-picture, is called the Captain, he communicates with his Navigators, responsible for state
estimation and path planning, and his First mates. These mates orchestrate and controller the Boatswains.
Actors who perform low-level dedicated tasks. Such as reading out a sensor signal, or controlling
the speed of an Archimedes screw. Because this framework is written in C++ it can be used on a multitude of different
controllers, ranging from dedicated specialized hardware, to \gls{acr-SBC} such as a \gls{acr-Rpi} or \gls{acr-BBB}.
This design descision has potential to greatly reduce the cost and shorten the lead time of future prototypes.

Validation of the controller is performed in a simulated environment. Chapter~\ref{chap:design validation} describes 
this in detail. It uses Project Chrono, a multi-body physics engine with specialized modules for vehicle simulation 
and terramechanics. However, tt lacks a module for sensor simulation. Something that is necessary in validating the 
performance of the \gls{acr-UKF}. Section~\ref{sec:sensor simulation} describes a custom written extension to Project
Chrono, called ``chrono\_sensors''. It allows for realistic sensor simulation, providing sensor signals to a 
controller, and is subject to noise, discretization, delay, and transformation. This extension was written for this 
project, but can also be integrated with other Project Chrono simulations, it is provided as an open-source project 
to the community under the MIT license.

A virtual scenario is created and described in Section~\ref{sec:simulation model}. This scenario is based on a 
use-case in which the crawler needs ton perform a maintenance dredging task in a small recreational harbor near 
Bruinisse in the Netherlands. The controller ``ohCaptain'' and ``chrono\_sensors'' are integrated in a simulation and
the same scenario is run three times. Section~\ref{sec:results} evaluates the different scenarios. The first scenario
can be considered as an ``analytical'' run. This is shows the path generated by the Captain and his Navigator with an
accurate location as basis. This scenario shows that the selected \gls{acr-topBCD} strategy does what its intended to
do. The whole harbor is covered in a seemingly logical way.

The second and third run are executed to show the difference between a \gls{acr-UKF} based controller and a simple
\gls{acr-PID} based one. The difference is rather pronounced, were the \gls{acr-PID} has a localization error that
keeps on growing, eventually letting the crawler think it is inside the body of water, while in reality it has
crossed the boundary and is now dredging on land. In real life unacceptably. The \gls{acr-UKF} based controller, has 
however managed to cover the whole soil bed.

\noindent The findings of this thesis are discussed in the conclusion, Chapter~\ref{chap:conclusion}.
